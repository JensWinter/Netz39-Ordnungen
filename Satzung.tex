\documentclass[a4paper,
               12pt,
               titlepage,
               parskip=half]{scrartcl}

\usepackage[left=3cm,right=2cm,top=2cm,bottom=2cm]{geometry} % Seitenränder
\usepackage[ngerman]{babel} % Spracheinstellungen

\usepackage{tabularx}
\usepackage[german, useregional]{datetime2}
\usepackage{eurosym}
\usepackage{enumitem}

\DTMsavedate{docdate}{2022-03-23}

% Hurenkinder und Schusterjungen verhindern
%\clubpenalty10000
%\widowpenalty10000
%\displaywidowpenalty=10000
\sloppy % macht ungefähren Blocksatz, wenn nichts anderes an Trennhilfen was nützt

\usepackage{tikz}

\def \logo {
	\begin{picture}(0,0)
		\put(-9, 2){\parbox{180mm}{
				%% Inline Logo
				\begin{tikzpicture}[
						y=0.8pt,
						x=0.8pt,
						yscale=-1,
						inner sep=0pt,
						outer sep=0pt,
						scale=0.2
					]
					\begin{scope}[shift={(-448.05262,-67.70133)}]
						\path[color=black,fill=black,even odd rule,line width=16.000pt]
						(534.1205,161.8816) -- (534.1205,172.2877) -- (581.3732,172.2877) .. controls
						(571.8452,181.8308) and (562.3036,191.3603) .. (552.7463,200.8741) --
						(565.3389,200.8741) .. controls (585.4511,200.8741) and (601.7400,217.2036) ..
						(601.7400,237.3158) .. controls (601.7400,257.4280) and (585.4511,273.7170) ..
						(565.3389,273.7170) .. controls (545.2267,273.7170) and (528.8972,257.4280) ..
						(528.8972,237.3158) -- (518.4911,237.3158) .. controls (518.4911,263.1743) and
						(539.4804,284.1231) .. (565.3389,284.1231) .. controls (591.1974,284.1231) and
						(612.1462,263.1743) .. (612.1462,237.3158) .. controls (612.1462,215.3233) and
						(596.9939,196.8355) .. (576.5548,191.8042) -- (606.5180,161.8816) -- cycle;
						\begin{scope}[cm={{-1.0,0.0,0.0,-1.0,(1130.6373,475.18252)}}]
							\path[color=black,fill=black,even odd rule,line width=16.000pt]
							(565.3389,208.6889) .. controls (549.5365,208.6889) and (536.7120,221.5134) ..
							(536.7120,237.3158) .. controls (536.7120,249.4467) and (544.2722,259.8031) ..
							(554.9328,263.9587) -- (554.9328,252.2164) .. controls (550.2090,248.9346) and
							(547.1181,243.5056) .. (547.1181,237.3158) .. controls (547.1181,227.2597) and
							(555.2828,219.0950) .. (565.3389,219.0950) .. controls (575.3950,219.0950) and
							(583.5192,227.2597) .. (583.5192,237.3158) .. controls (583.5192,243.4955) and
							(580.4441,248.9328) .. (575.7450,252.2164) -- (575.7450,263.9587) .. controls
							(586.3945,259.8031) and (593.9253,249.4467) .. (593.9253,237.3158) .. controls
							(593.9253,221.5134) and (581.1413,208.6889) .. (565.3389,208.6889) -- cycle;
							\path[color=black,fill=black,even odd rule,line width=16.000pt]
							(565.3389,232.0925) .. controls (562.4657,232.0925) and (560.1156,234.4426) ..
							(560.1156,237.3158) -- (560.1156,268.4937) -- (570.5217,268.4937) --
							(570.5217,237.3158) .. controls (570.5217,234.4426) and (568.2121,232.0925) ..
							(565.3389,232.0925) -- cycle;
						\end{scope}
						\path[fill=black] (630.8529,149.4914) .. controls (625.1306,149.5458) and
						(620.4890,150.3447) .. (619.3535,151.0301) .. controls (617.0825,152.4009) and
						(616.5461,155.6877) .. (617.2480,156.4558) .. controls (617.9500,157.2237) and
						(628.6785,164.1143) .. (633.3633,166.3761) .. controls (645.9778,172.4665) and
						(632.0756,195.7720) .. (621.2161,191.6018) .. controls (613.5199,188.6464) and
						(604.7892,179.8446) .. (603.3192,179.7785) .. controls (601.8492,179.7137) and
						(599.9557,180.7048) .. (599.4726,183.2607) .. controls (598.9895,185.8166) and
						(602.4810,198.1576) .. (611.2554,207.7576) .. controls (621.7846,219.2778) and
						(638.1817,206.5491) .. (638.1817,217.0704) --
						(638.1817,265.9427)arc(179.961:0.039:13.018) -- (664.2172,203.5060) --
						(664.2172,188.9294) .. controls (664.2172,183.4623) and (670.9201,161.5406) ..
						(648.7498,152.3663) .. controls (643.3754,150.1424) and (636.5752,149.4370) ..
						(630.8529,149.4914) -- cycle;
					\end{scope}
					\path[color=black,fill=black,even odd rule,line width=8.333pt]
					(150.0000,23.1213) .. controls (79.9291,23.1213) and (23.1213,79.9291) ..
					(23.1213,150.0000) .. controls (23.1213,220.0708) and (79.9291,276.8787) ..
					(150.0000,276.8787) .. controls (220.0708,276.8787) and (276.8787,220.0708) ..
					(276.8787,150.0000) .. controls (276.8787,79.9291) and (220.0708,23.1213) ..
					(150.0000,23.1213) -- cycle(150.0000,40.5068) .. controls (210.4603,40.5068)
					and (259.4540,89.5397) .. (259.4540,150.0000) .. controls (259.4540,210.4603)
					and (210.4603,259.4540) .. (150.0000,259.4540) .. controls (89.5278,259.8428)
					and (41.3547,203.5210) .. (40.5068,150.0000) .. controls (40.5068,89.5397) and
					(89.5397,40.5068) .. (150.0001,40.5068) -- cycle;
				\end{tikzpicture}
				%% Logo End
		}}
	\end{picture}
}

\usepackage[colorlinks=true,linkcolor=black,citecolor=black,urlcolor=black,breaklinks=true]{hyperref}

\setlength{\headheight}{0.25\baselineskip}

\begin{document}

\logo 
\hspace{3em}
\begin{minipage}{\textwidth}
	{\Large \textbf{Satzung}}\\
	des \textit{Netz39 e.~V. -- Hackerspace Magdeburg}\\
	in der Fassung von \DTMusedate{docdate}
\end{minipage}

\vspace{2.0em}

{\large \textbf{Präambel}}\\\\
Im Folgenden finden sich Satzungen und Ordnungen des Vereins Netz39 auf dem Stand vom
\DTMusedate{docdate}. An Stellen, an denen schriftliche Kommunikation gefordert wird, ist E-Mail stets mit eingeschlossen.

\vspace{1.5em}

{\large \textbf{Vereinssatzung}}\\[1em]
%
\textbf{§1 Name und Sitz des Vereins; Geschäftsjahr}
{\small
\begin{enumerate}
	\item Der Verein führt den Namen \glqq Netz39\grqq\ -- im Folgenden \glqq Verein\grqq\  genannt.
	\item Der Verein hat seinen Sitz in Magdeburg. Er soll in das Vereinsregister eingetragen werden und führt anschließend den Zusatz \glqq e.V.\grqq .
	\item Das Geschäftsjahr ist das Kalenderjahr.
\end{enumerate}
}

\vspace{1.0em}

\textbf{§2 Vereinszweck}
{\small 
\begin{enumerate}
	\item Der Zweck des Vereins ist die Förderung von Wissenschaft, Forschung, Kunst, Kultur, Meinungs- und Wissensaustausch. Insbesondere sollen Bereiche der Informations- und Kommunikationsmedien, informationsverarbeitende Technologien, deren angrenzende Fachgebiete, handwerkliches Geschick, autodidaktisches Lernen, Erwachsenen- und Jugendbildung gefördert werden. Auf diese Weise sollen Kultur, Computerkunst, Bildung und Wissenschaft in neuen und bestehenden Formen ermöglicht werden. Netz39 schafft einen Anlaufpunkt für den technischen, gesellschaftlichen und kulturellen Austausch im Bereich informationsverarbeitender Technologien.
	\item Für die Erfüllung dieser satzungsgemäßen Zwecke sollen geeignete Mittel durch Beiträge/Umlagen, Spenden, Zuschüsse und sonstige Zuwendungen eingesetzt werden.
	\item Der Verein verfolgt ausschließlich und unmittelbar gemeinnützige Zwecke im Sinne des Abschnitts \glqq Steuerbegünstigte Zwecke\grqq\ der Abgabenordnung 1977 (§51 ff AO) in der jeweils gültigen Fassung. Der Verein ist selbstlos tätig. Er verfolgt nicht in erster Linie eigenwirtschaftliche Zwecke.
	\item Mittel des Vereins dürfen nur für die satzungsgemäßen Zwecke verwendet werden.
	\item Die Mitglieder erhalten in ihrer Eigenschaft als Mitglieder keine Zuwendungen aus Mitteln des Vereins. Es darf keine Person durch Ausgaben, die den Zwecken des Vereins fremd sind, oder durch unverhältnismäßig hohe Vergütungen begünstigt werden.
\end{enumerate}
}

\newpage

\textbf{§3 Mitgliedschaft}
{\small
\begin{enumerate}
	\item Der Verein besteht aus aktiven Mitgliedern (§3 Abs. 2) und Fördermitgliedern (§3 Abs. 3), die bereit sind, sich für die Erreichung der Vereinszwecke einzusetzen. Die Beitrittserklärung erfolgt schriftlich gemäß §4 gegenüber dem Vorstand.
	\item Die aktive Mitgliedschaft kann von jeder natürlichen Person erworben werden, die sich zum Vereinszweck bekennt und durch aktive Mitarbeit einen regelmäßigen Beitrag leistet.
	\item Die Fördermitgliedschaft kann von natürlichen und juristischen Person erworben werden, die den Verein bei der Erreichung seines Vereinszwecks unterstützen will.
	\item Über die Aufnahme als Mitglied entscheidet der Vorstand. Will der Vorstand einen Aufnahmeantrag ablehnen, so legt er ihn der nächsten Mitgliederversammlung vor. Diese entscheidet endgültig.
\end{enumerate}
}

\vspace{1.0em}

\textbf{§4 Beginn und Ende der Mitgliedschaft}
{\small
	\begin{enumerate}
		\item Die Mitgliedschaft muss gegenüber dem Vorstand schriftlich beantragt werden. Über den schriftlichen Aufnahmeantrag wird entsprechend §3 entschieden. Der Vorstand ist nicht verpflichtet, dem Antragsteller Ablehnungsgründe mitzuteilen.
		\item Die Mitgliedschaft beginnt mit der Aushändigung einer entsprechenden Bestätigung durch ein Vorstandsmitglied.
		\item Die Mitgliedschaft endet
		\begin{itemize}
			\item durch freiwillige Beendigung mit zweiwöchiger Frist zum Ende des Quartals durch schriftliche Erklärung gegenüber dem Vorstand;
			\item bei natürlichen Personen durch Tod;
			\item bei juristischen Personen mit Abschluss der Liquidation;
			\item bei Eintritt der Insolvenz des Vereins;
			\item durch Ausschluss gemäß §6.
		\end{itemize}
		\item Bei Beendigung der Mitgliedschaft, gleich aus welchem Grund, erlöschen alle Ansprüche aus dem Mitgliedsverhältnis. Eine Rückgewähr von Beiträgen, Spenden oder sonstigen Unterstützungsleistungen ist grundsätzlich ausgeschlossen.
		\item Der Anspruch des Vereins auf rückständige Beitragsforderungen bleibt von einem Ende der Mitgliedschaft unberührt.
	\end{enumerate}
}

\vspace{1.0em}

\textbf{§5 Rechte und Pflichten der Mitglieder}
{\small
	\begin{enumerate}
		\item Die aktiven Mitglieder sind berechtigt, an allen angebotenen Veranstaltungen des Vereins teilzunehmen. Sie haben darüber das Recht, gegenüber dem Vorstand und der Mitgliederversammlung Anträge zu stellen und an Abstimmungen teilzunehmen.
		\item Fördermitglieder haben das Recht, Vorschläge zu Aktivitäten des Vereins zu machen und Informationen über die Verwendung der Förderbeiträge zu erhalten.
		\item Die Mitglieder sind verpflichtet, den Verein und Vereinszweck - auch in der Öffentlichkeit - in ordnungsgemäßer Weise zu unterstützen.
		\item Der Verein erhebt einen Mitgliedsbeitrag, zu dessen Zahlung die Mitglieder verpflichtet sind. Näheres regelt eine Beitragsordnung, die von der Mitgliederversammlung beschlossen wird.
	\end{enumerate}
}

\newpage

\textbf{§6 Ausschluss aus dem Verein}
{\small
	\begin{enumerate}
		\item Der Ausschluss eines Mitgliedes mit sofortiger Wirkung und aus wichtigem Grund kann dann ausgesprochen werden, wenn das Mitglied in grober Weise dem Zwecke, der Satzung, den Zielen oder der Ordnung des Vereins zuwider handelt oder das Ansehen des Vereins in der Öffentlichkeit in grober Weise schädigt.
		\item Über den Ausschluss entscheidet der Vorstand in einfacher Stimmmehrheit.
		\item Dem Mitglied ist unter Fristsetzung von zwei Wochen Gelegenheit zu geben, sich vor dem Vereinsausschluss zu den Vorwürfen zu äußern. Legt das Mitglied gegen den Ausschluss Widerspruch beim Vorstand ein, so entscheidet die Mitgliederversammlung endgültig über den Ausschluss.
	\end{enumerate}
}

\vspace{1.0em}

\textbf{§7 Organe des Vereins}
{\small
	\begin{enumerate}
		\item Organe des Vereins sind
		\begin{itemize}
			\item die Mitgliederversammlung;
			\item der Vorstand.
		\end{itemize}
	    \item Einem Organ des Vereins können nur aktive Mitglieder des Vereins angehören.
	\end{enumerate}
}

\vspace{1.0em}

\textbf{§8 Die Mitgliederversammlung}
{\small
	\begin{enumerate}
		\item Die Mitgliederversammlung ist das oberste Beschlussorgan des Vereins. Ihr obliegen alle Entscheidungen, die nicht durch die Satzung oder die Beitrags- beziehungsweise Geschäftsordnung einem anderen Organ übertragen wurden.
		\item Beschlüsse werden von der Mitgliederversammlung durch öffentliche Abstimmung getroffen. Auf Wunsch eines Mitglieds ist geheim abzustimmen.
		\item Eine ordentliche Mitgliederversammlung wird vom Vorstand des Vereins nach Bedarf, mindestens aber einmal im Geschäftsjahr, einberufen. Die Einladung erfolgt 14 Tage vorher schriftlich durch den Vorstand mit Bekanntgabe der vorläufig festgesetzten Tagesordnung an die dem Verein zuletzt bekannte Adresse des Mitglieds.
		\item Der Vorstand hat eine außerordentliche Mitgliederversammlung unverzüglich einzuberufen,
		wenn die Einberufung von mindestens einem Drittel der stimmberechtigten Vereinsmitglieder schriftlich unter Angabe des Zwecks und der Gründe vom Vorstand verlangt wird.
		\item Anträge der Mitglieder zur Tagesordnung sind spätestens eine Woche vor der Mitgliederversammlung beim Vorstand schriftlich einzureichen. Nachträglich eingereichte Tagesordnungspunkte müssen den Mitgliedern rechtzeitig vor Beginn der Mitgliederversammlung mitgeteilt werden.
		\item Spätere Anträge - auch während der Mitgliederversammlung gestellte Anträge - müssen auf die Tagesordnung gesetzt werden, wenn in der Mitgliederversammlung die Mehrheit der erschienenen stimmberechtigten Mitglieder der Behandlung der Anträge zustimmt (Dringlichkeitsanträge).
		\item Die Mitgliederversammlung wird von der*dem Vorsitzenden oder der Stellvertretung geleitet. Auf Vorschlag der*des Vorsitzenden kann die Mitgliederversammlung eine besondere Versammlungsleitung bestimmen.
		\item Beschlüsse der Mitgliederversammlung werden in einem Protokoll innerhalb von zwei Wochen nach der Mitgliederversammlung niedergelegt und von zwei Vorstandsmitgliedern unterzeichnet. Abschriften des Protokolls werden den Mitgliedern schriftlich zugestellt.
		\item Stimmberechtigt sind alle aktiven Mitglieder des Vereins. Jedes stimmberechtigte Mitglied hat genau eine Stimme, die nur persönlich ausgeübt werden darf.
		\item Die Mitgliederversammlung ist beschlussfähig, wenn die Einladung ordnungsgemäß erfolgt ist und bei der Mitgliederversammlung mindestens 20\% der aktiven Mitglieder des Vereins, mindestens aber fünf aktive Mitglieder anwesend sind. Im Falle einer nicht beschlussfähigen Mitgliederversammlung ist durch den Vorstand innerhalb von zwei Wochen eine Mitgliederversammlung ordnungsgemäß einzuberufen, die in jedem Fall beschlussfähig ist. Dies muss in der Einladung explizit vermerkt sein.
		\item Die Mitgliederversammlung fasst ihre Beschlüsse mit einfacher Mehrheit. Stimmenthaltungen bleiben außer Betracht. Bei Stimmgleichheit gilt der gestellte Antrag als abgelehnt.
		\item Abstimmungen der Mitgliederversammlungen erfolgen offen durch Handheben oder Zuruf.
	\end{enumerate}
}

\vspace{1.0em}

\textbf{§9 Satzungs- und Ordnungsänderungen}
{\small
	\begin{enumerate}
		\item Über Satzungs- und Beitragsordnungsänderungen kann die Mitgliederversammlung abstimmen, wenn auf diesen Tagesordnungspunkt hingewiesen wurde und der Einladung sowohl der bisherige als auch der vorgesehene neue Text beigefügt wurden.
		\item Für die Satzungs- und Ordnungsänderungen ist eine Mehrheit von zwei Dritteln in der Mitgliederversammlung erforderlich.
		\item Satzungsänderungen, die von Aufsichts-, Gerichts- oder Finanzbehörden aus formalen Gründen verlangt werden, kann der Vorstand von sich aus vornehmen. Diese Satzungsänderungen müssen auf der nächsten Mitgliederversammlung mitgeteilt werden.
	\end{enumerate}
}

\vspace{1.0em}

\textbf{§10 Der Vorstand}
{\small
	\begin{enumerate}
		\item Der Vorstand setzt sich zusammen aus der*dem Vorsitzenden, der Stellvertretung und der*dem Schatzmeister*in.
		\item Die Vorstandsmitglieder werden von der Mitgliederversammlung aus den aktiven Mitgliedern für die Dauer von einem Jahr gewählt. Die unbegrenzte Wiederwahl von Vorstandsmitgliedern ist zulässig. Nach Fristablauf bleiben die Vorstandsmitglieder bis zum Antritt ihrer Nachfolger*innen im Amt.
		\item Der Vorstand leitet verantwortlich die Vereinsarbeit. Er kann sich eine Geschäftsordnung geben und besondere Aufgaben unter seinen Mitgliedern verteilen oder Ausschüsse für die Bearbeitung oder Vorbereitung einsetzen.
		\item Der Verein wird im Sinne des §26 BGB von der*dem Vorsitzenden und von der Stellvertretung jeweils einzeln gerichtlich und außergerichtlich vertreten. Die*Der Vorsitzende und die Stellvertretung sind von den Beschränkungen des §181 BGB befreit.
		\begin{enumerate}[label=\arabic{enumi}\alph*.,leftmargin=0.0in]
			\item Die folgenden Beschlüsse können nicht durch den Vorstand gefällt werden:\\
			- Entscheidungen zur Kündigung oder Aufnahme eines Mietverhältnisses für die Räumlichkeiten des Vereins.\\\\
			Soll die Mitgliederversammlung über einen der oben genannten Punkte entscheiden, so
			ist das auf der Einladung als Tagesordnungspunkt kenntlich zu machen.
	    \end{enumerate}
    	\item Der Vorstand beschließt mit einfacher Stimmmehrheit. Der Vorstand ist beschlussfähig, wenn mindestens zwei Mitglieder anwesend sind oder schriftlich zustimmen. Bei Stimmgleichheit gilt der Antrag als abgelehnt. Beschlüsse des Vorstands werden in einem Sitzungsprotokoll niedergelegt und von mindestens zwei Vorstandsmitgliedern unterzeichnet.
    	\item Scheidet ein Vorstandsmitglied vor Ablauf seiner Wahlzeit aus, ist der Vorstand berechtigt, ein kommissarisches Mitglied zu berufen. Auf diese Weise bestimmte Vorstandsmitglieder bleiben bis zur nächsten Mitgliederversammlung im Amt.
    	\item Die*Der Vorsitzende zeichnet verantwortlich für die Berichtsfähigkeit gegenüber der Aufsichtsbehörde.
    	\item Die Vorstandsmitglieder sind grundsätzlich ehrenamtlich tätig; sie haben Anspruch auf Erstattung notwendiger Auslagen, sofern deren Rahmen in der Geschäftsordnung festgelegt wird.
    	\item Die Mitgliederversammlung stimmt über die Entlastung des Vorstands ab.
	\end{enumerate}
}

\vspace{1.0em}

\textbf{§11 Kassenprüfung}
{\small
	\begin{enumerate}
		\item Über die Jahresmitgliederversammlung sind zwei Kassenprüfende für die Dauer von einem Jahr zu wählen. Diese dürfen nicht Mitglied des Vorstands oder eines vom Vorstand berufenen Gremiums sein.
		\item Die Kassenprüfenden haben die Aufgabe, Rechnungsbelege sowie deren ordnungsgemäße Verbuchung und die Mittelverwendung zu prüfen und dabei insbesondere die satzungsgemäße und steuerlich korrekte Mittelverwendung festzustellen. Die Prüfung erstreckt sich nicht auf die Zweckmäßigkeit der vom Vorstand getätigten Ausgaben.
		\item Die Kassenprüfenden haben die Mitgliederversammlung über das Ergebnis ihrer Kassenprüfung zu unterrichten und werden auf Grundlage des Berichts von der Mitgliederversammlung entlastet.
	\end{enumerate}
}

\vspace{1.0em}

\textbf{§12 Veranstaltungen und Projekte}
{\small
	\begin{enumerate}
		\item Der Verein darf von den Teilnehmenden einer Veranstaltung oder eines Projekts einen jeweils vom Vorstand festzusetzenden Unkostenbeitrag erheben.
		\item Die Veranstaltungen und Projekte dürfen von Personen geleitet werden, die nicht dem Verein angehören. Die Leitung darf eine gewerbliche Dienstleistung sein, die aus der Vereinskasse vergütet wird.
	\end{enumerate}
}

\vspace{1.0em}

\textbf{§13 Übertragung von Aufgaben}
{\small
	\begin{enumerate}
		\item Der Vorstand ist berechtigt, bestimmte Aufgaben zur Erfüllung des Vereinszwecks auf
		andere Personen, insbesondere auf Veranstaltungs- und Projektleitende, zu übertragen.
		\item Die Übertragung darf schriftlich oder mündlich erfolgen.
	\end{enumerate}
}

\vspace{1.0em}

\textbf{§14 Auflösung des Vereins}
{\small
	\begin{enumerate}
		\item Die Auflösung des Vereins muss von der Mitgliederversammlung mit einer Mehrheit von zwei Dritteln beschlossen werden.
		\item Die Abstimmung ist nur möglich, wenn auf der Einladung zur Mitgliederversammlung als einziger Tagesordnungspunkt die Auflösung des Vereins angekündigt wurde.
		\item Bei Auflösung des Vereins oder bei Wegfall seiner steuerbegünstigten Zwecke fällt das Vereinsvermögen an digitalcourage e.V., der es ausschließlich und unmittelbar zu steuerbegünstigten Zwecken zu verwenden hat.
	\end{enumerate}
}

\newpage

\textbf{§15 Mitgliedschaft in anderen Vereinen}
{\small
	\begin{enumerate}
		\item Der Verein darf Mitglied in anderen Vereinen werden.
		\item Die Mitgliederversammlung entscheidet über den Beitritt bzw. Austritt aus anderen Vereinen.
		\item Diese Entscheidung kann auch vom Vorstand gefällt werden und muss in der nächsten Mitgliederversammlung bestätigt werden. Falls die Mitgliederversammlung gegen die vom Vorstand getroffene Entscheidung stimmt, erfolgt der Austritt aus dem Verein zum nächstmöglichen Termin.
	\end{enumerate}
}

\vspace{1.0em}

\textbf{§16 Schriftform}
{\small
	\begin{enumerate}
		\item Schriftliche Erklärungen im Sinne dieser Satzung können auch elektronische Dokumente sein.
		\item Die Geschäftsordnung bestimmt Anforderungen, Zustellwege und Zuordnung derartiger Dokumente.
	\end{enumerate}
}



\end{document}

